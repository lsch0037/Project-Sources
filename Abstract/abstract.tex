\documentclass[11pt, a4paper]{article}

\usepackage{hyperref}
\usepackage{url}
\usepackage{graphicx}
\graphicspath{{./images/}}

\begin{document}
\begin{flushleft}

% Title
Procedurally Generating Interesting Structures in 3D Video Games\linebreak

% Overview
%? What is level design.
%? Simplify level design.
%?Allows the process to be automated.

\paragraph*{}
The level design phase in video games is laborious process, requiring designers to place object manually within a 3D environment.
This work seeks to automate this process by providing a solution that uses procedural generation methods to populate game levels with structures in a cohesive and interesting manner.


% Implementation
%? DSL and PCG Library were created.
%? Explain the four layers and give a justification for each.
%? The input indentifies the object and their relations to each other and composites a program in the DSL.
%? The Lanugage Layer uses language constructs such as loops, if-statements and functions which are finally evaluated into a complex expression of primitive geometry and operations between them defining all the structures.
%? The PCG Tree is then evaluated using a set of algorithms to create a buffer representation of each blocks position in the final game.
%? Finally the buffer can be written to the game world using an interface for minecraft provided by the [] python server and [] plugin.

\paragraph*{}
Minecraft was chosen to be the target 3D environment of this project on account of its widespread popularity and its voxel-based graphics allowing expressive manipulation of the scenery.
A custom tool was developed to parse an input paragraph describing a scene, such as "A tall oak tree near a brick house, on top of a mountain," and generate the appropriate geometry accordingly.
To this end, a Domain Specific Lanugage was created to define structures in terms of geometric primitives, other structures, and operations on and between them.
A dedicated library was then implemented to enable to computation of geometry in terms of the positions and materials of individual voxels.
Finally, the Spigot Server application [1] was employed, in combination with the MCPI Python Plugin [2], to generate the final scene in the game's 3D environment.

% Evaluation
\paragraph*{}
The system was tested given a number of diverse and complex prompts and the resulting output was evaluated according to a number of objective and subjective criteria.
% The system was able to generate levels that were accurate interpretations of the given input and that corresponded to the user's expectations.

\end{flushleft}
\end{document}