\documentclass{fict}

\addbibresource{references.bib}

\input{content/glossary}
\input{content/acronyms}

\usepackage{hyperref}
\usepackage{url}
\usepackage{graphicx}
\graphicspath{{./images/}}

\title{Procedurally Generating Interesting Structures in 3D Video Games}
\author{Lukas Schimpf}
\supervisor{Dr.\ Keith Bugeija}
\cosupervisor{Dr Sandro Spina}
\degreename{Something}
\titledate{June 2025}

\begin{document}

\begin{flushleft}
Content
\pagebreak

\section*{Abstract}

    \subsection*{}
    The level design phase in video games represents a critical and labour-intensive process, requiring designers to meticulously place objects manually within a three-dimensionsal environment.
    Traditionally, this task has been performed manually, often requiring significant time and effort from designers.
    This study seeks to streamline the level design process by proposing an innovative solution that employs procedural generation techniques, automating the population of game levels with structures in a coherent, interesting and visually appealing manner.

    % Implementation
    %? DSL and PCG Library were created.
    %? Explain the four layers and give a justification for each.
    %? The input indentifies the object and their relations to each other and composites a program in the DSL.
    %? The Lanugage Layer uses language constructs such as loops, if-statements and functions which are finally evaluated into a complex expression of primitive geometry and operations between them defining all the structures.
    %? The PCG Tree is then evaluated using a set of algorithms to create a buffer representation of each blocks position in the final game.
    %? Finally the buffer can be written to the game world using an interface for minecraft provided by the [] python server and [] plugin.

    \subsection*{}
    Minecraft was chosen to be the target environment of this project on account of its widespread popularity and its voxel-based graphics allowing expressive manipulation of the scenery.
    To achieve the desired automation, a custom tool was developed, capable of parsing an input paragraph describing a scene, (e.g., "A tall oak tree near a brick house, on top of a mountain"). The tool interprets this description in order to generate the appropriate geometry within the Minecraft environment.
    To this end, a Domain Specific Lanugage (DSL) was devised, which allows for the definition of structures in terms of geometric primitives, other structures, and operations applied to and between them.
    This DSL not only streamlines the process of creating new structures but also provides a robust framework for generating complex and diverse environments.


    \subsection*{}
    A dedicated library was then implemented to enable to computation of geometry in terms of the positions and materials of individual voxels.
    This library enables the efficient generation of intricate structures and landscapes within the Minecraft world, reducing the manual workload for designers while maintaining a high level of quality and detail.
    In order to integrate the generated structures and landscapes into the game's three-dimensional environment, the Spigot Server application was employed, in conjunction with the MCPI Python Plugin. This combination allowed for seamless communication between the procedural generation tool and the Minecraft environment, ensuring that the generated scenes were accurately represented in the game world.


    % Evaluation
    \subsection*{}
    The system was tested using a variety prompts, simulating real-world use cases and challenging the system's capabilities. The resulting output was evaluated according to a range of objective and subjective criteria, including the fidelity of the generated structures to their descriptions, the visual coherence and appeal of the scene, and the efficiency of the genereation process

    % Results
    \subsection*{}
    The evaluation shows that by automating the creation of immersive and engaging environments, designers can dedicate their focus to refining gameplay elements, optimizing performance, and enhancing the overall player experience. Moreover, the procedural generation techniques employed in the system contribute to a more dynamic and replayable gaming experience, as the generated levels exhibit greater variability and uniqueness.

    \pagebreak


\section*{Problem Definition}
\begin{enumerate}
    \item What is level design.
    \item How can it be improved, automated.
    \item What are some criteria that are important for the system (evaluation criteria).
\end{enumerate}

\section*{Background and Literature Review}
\begin{enumerate}
    \item what is level design.
    \item What is procedural generation.
    \item What is Natural language processing.
    \item What are domain specific languages.
    \item How are 3d object represented in video games.
    \item Some content about vectors and matrices and their relation to 3d games.
    \item Minecraft.
\end{enumerate}

\section*{Methodology / Solution}
\begin{enumerate}
    \item How is the project split up conceptually into layers.
    \item Give justification for each layer.
\end{enumerate}

\subsection*{Layer 1}

\subsection*{Layer 2}

\subsection*{Layer 3}

\subsection*{Layer 4}


\section*{Evaluation and Results}
\begin{enumerate}
    \item Case Studies.
    \item Comment on the accuracy.
    \item Performance.
    \item Comment on the limitations of the system.
\end{enumerate}
% TODO FIND EVALUATION CRITERIA

\section*{Future Work}
\begin{enumerate}
    \item Natural language processing.
    \item Performance optimisation.
    \item AI to learn the definition of structures.
\end{enumerate}

\section*{Conclusion}
\begin{enumerate}
    \item What was achieved and to what degree?
    \item What are some shortcomings of the implementation or what further areas remain to be explored?
\end{enumerate}
    
\end{flushleft}
\end{document}